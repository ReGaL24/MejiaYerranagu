\section{Implementation, Integration and Test Plan}
\subsection{Overview}
This chapter outlines the implementation, integration, and testing strategies for the BBP system with the primary goal being the development of mobile centered architecture with microservices across 4 tier architecture. It details the development environment, tools, and methodologies that will be employed to ensure a robust and reliable application. The chapter also describes the testing plan, including unit tests, integration tests, system tests, and user acceptance tests to validate the functionality and performance of the system.

The system can be broken down into main components like the mobile application, backend microservices, database, and external service integrations. For the process of implementing, integrating, and testing these components, Bottom-up approach with Thread strategy will be used. 

\subsection{Implementation Plan}
The implementation of the BBP system will be carried out in iterative phases, following combined bottom-up and thread strategy to ensure that each component is developed, integrated, and tested thoroughly before moving on to the next one. This implementation further facilitates bug tracking by allowing incremental development and testing of each component. 
\begin{itemize}
    \item Thread Strategy focuses on implementing and testing end to end functionality of vital features that spans across multiple components. One such example is the trip recording and weather enrichment feature that involves implementation and testing of mobile application, backend microservices, database interactions, and external weather API integration.
    \item Bottom-up approach will be used to implement and test individual components starting from the lowest level such as databases, Device sensor adapters before being integrated into higher-level services such as Trip recording services. This approach ensures that each low level dependency if fully functional before being integrated into more complex higher level services.
\end{itemize} 
The implementation will be carried out in the following phases:
\begin{itemize}
    
    \item \textbf{Phase 1: Foundation}
    This phase focuses on setting up the development environment, version control systems, and continuous integration pipelines. beginning with design and implementation of data schemas, and core microservices such as user authentication services, Trip recording services- basic CRUD and API Gateway will be developed and tested. Along with these a basic UI for user registration and login with the ability to view empty trip list will be implemented

    \item \textbf{Phase 2: Core Functionality}
    In this phase, the primary functionality of the mobile application will be implemented such as device sensory adapters in client side for GPS and motion sensor data. In addition to this trip recording will get enhanced functionality with operations such as starting, pausing, resuming, completing a trip for computing trip statistics. Persist trip and statistics in the database and expose them to the corresponding API and add basic user interface components. Each feature will be developed using the bottom-up approach, ensuring that all underlying services are functional before integrating them into the mobile application.

    \item \textbf{Phase 3: Advanced Features}
    This phase will focus on implementing advanced features such as data visualization, Path Management Service, Path Scoring Service and the interaction between them. Integration of Geocoding services to aid path management for Origin-Destination queries and integration tests will be conducted to ensure that all components work seamlessly together.

    \item \textbf{Phase 4: Final Integration and Testing}
    The final phase involves implementation of weather integration services and its interaction with external agents, along with implementation of notification services to aid in delivering push notifications. Subsequently, event bus is implemented for asynchronous communication between services. Comprehensive system testing, including performance testing, security testing, and user acceptance testing is to be carried out. Any identified issues will be addressed before the final deployment of the application.    
\end{itemize}

\subsection{Integration Plan}
The integration of the BBP system components will be carried out in a systematic manner to ensure that it happens at service boundaries rather than database level. This approach minimizes the risk of integration issues and ensures that each component can communicate effectively with others. The integration plan follows the bottom-up approach with thread based strategy, starting with the integration of low-level components and gradually moving up to higher-level services. Testing in mobile environments presents unique challenges due to the diversity of devices, operating systems, and network conditions. Therefore, the testing plan includes strategies to address these challenges, such as testing on a range of devices and emulators, as well as simulating different network conditions to ensure robust performance across various scenarios.
% Low level components are first tested in isolation within microservices to deem stable before being integrated into higher level services. For example, the database interactions are tested within each microservice before integrating them into the API Gateway. Once low level components are stable, higher level services are integrated using thread based strategy to ensure that end to end functionality is validated across multiple components. For instance, the trip recording feature involves integration of mobile application, Trip recording service, database and external weather API.
Integration will occur across the three primary interfaces:
\begin{itemize}
    \item[-] \textbf{Synchronous Communication}: Communication between the mobile layer and the backend are fulfilled via API Gateway with the use of RESTful APIs over HTTPS. Each microservice exposes its functionality through well-defined endpoints, allowing the mobile application to interact with them seamlessly. API Gateway handles request routing, authentication, and rate limiting to ensure secure, efficient communication and decoupling from the microservices.
    \item[-] \textbf{Asynchronous Communication}: For inter-service communication within the backend, an event-driven architecture is employed using a message broker, in the case of BBP system, through RabbitMQ. This allows services to publish and subscribe to events, enabling loose coupling and scalability. For example, when a trip is completed, the Trip Recording Service can publish an event called \verb|Trip.Completed| that the Weather Integration Service subscribes to for weather enrichment. Integration testing at this phase can look into apt event publishing and consumption across services.
    \item[-] \textbf{Data Persistence}: The BBP system uses a database per service for data persistence, with each microservice managing its own schema enriched database. Integration at this level involves ensuring data consistency before being exposed to services APIs. For example, the Trip Recording Service must ensure that trip data is correctly stored and retrieved from its database before being made available through its API endpoints.
\end{itemize}

Services are to be integrated in an incremental fashion, starting with the core services such as User Authentication Service and Trip Recording Service, followed by Path Management Service and its peer Path Scoring Service, Weather Integration Service, and Notification Service. Each integration step is followed by rigorous testing to validate the interactions between services and ensure that data flows correctly across the system.
Once the microsrvices are integrated, API Gateway is to be configured with routing rules for each microservice, along with security policies such as authentication and rate limiting to ensure secure and efficient communication between the mobile application and backend services. Caching strategies are also implemented at the API Gateway level to enhance performance and reduce latency for frequently accessed data. 
As far as integration of client side mobile application is concerned, initially mock servers are used to simulate the services which can gradually be replaced with real API endpoints as the backend services get integrated and stabilized. This approach allows for parallel development of the mobile application and backend services, reducing overall development time. Testing can be carried out on iOS emulators and Android ones, along with testing for offline caching. In addition to these and similarly integration of graceful degradation of BBP system with Redis for caching needs to be Implemented.

Further to main high integration stabillity, A continuous integration (CI) pipeline is established to automate the build with static code analysis checking for syntax errors and standard code violations including readability and naming convention to name a few , unit testing to be carried out to test logic within services and deployment of these successful builds with the help of containers. 
In addition to these, since the devices that run the BBP is diverse in terms of hardware capabilities, operating systems and also with respect to network conditions, Integration planning must include strategies to address these pitfalls. 
\subsection{System Testing}

The system testing plan aims to validate the behavior of the fully integrated BBP system under realistic usage conditions, once all the components have been integrated. We propose different strategies to guarantee that the system behaves as expected under different circumstances and case scenarios, which are derived from system functional requirements and use cases and focus on verifying complete realistic workflows along all the layers in the 4 tier proposed architecture, including mobile interaction patterns, stateful user actions, long-running activities, and dependencies on external services.

The functional testing strategy verifies that end-to-end user workflows behave according to requirements, validating core functionality such as trip recording, path discovery, manual and automatic path contribution, and data visualization. We give special attention to the case of stateful conditions such as trip lifecycle management and path publication and visibility rules, for which we can follow defined constraints that allow us to assess a correct management of data. For example, automatically sensed path data remains private by default until a user explicitly confirms its publication. We also relate functional testing to the differentiation among registered and unregistered users and the proper enforcement of access restrictions.

Performance testing evaluates the responsiveness of the system under expected conditions. Since we have defined expected completion times for specific service interactions, the goal is to ensure that those metrics are being held such that system remains responsive during normal usage. Load testing extends this strategy by applying increased usage conditions like concurrent access from multiple users, so we can assess and guarantee that BBP system still can sustain a higher service demand without functional degradation. Furthermore, stress testing examines the robustness of the system under extreme conditions such as eventual spikes in requests or region unavailability, to validate gracefull degradation strategies, resilience and data recovery mechanisms.

On the other hand, usability testing focuses more on the presentation layer of the architecture, assessing parameters like intuitiveness and responsiveness of the system touch-based user interface for mobile devices. This strategy handles the verification of core interaction flows, such as starting and stopping trip recording, reviewing path publication, requesting routes between origin and destination, navigating through maps, etc. Usability testing complements the current system testing strategy by aiming to ensure a smooth user experience rather than focusing on architectural correctness.

Security and privacy testing is included in the testing plan to verify that the system adequately protects data and is resilient to common attacks. We address testing for well-known threats such as injection attacks and cross-site scripting, as well as the correct handling of user sensitive data like location. These tests ensure that GPS traces and user information is stored, transmitted and accessed complying with privacy requirements.

One important aspect of the complete system testing strategy is the validation of reliability and fault-tolerance mechanisms that happens at runtime. For example, offline data handling with the help of caching adapters should be assessed to guarantee synchronization once connectivity is restored. On the other hand, expected retry strategies for interactions with external services are also expected to be gracefully handled. These scenarios are critical to ensuring correct system behavior in mobile environments where network conditions may be unstable. External services, including geocoding and weather providers, are replaced by simulated endpoints during testing to ensure repeatability and to allow controlled evaluation of scenarios such as timeouts and service unavailability. Test data is isolated across test runs, and persistent storage is reset to a known state to avoid cross-test interference. 

Finally, acceptance testing is performed to validate that the system meets expectations and regulatory requirements. This includes compliance testing against data protection regulations such as GDPR. Also, Alpha and Beta testing phases are included for cases that involve real-life sensor data to confirm that the system behaves correctly under realistic operating conditions that cannot be emulated accurately on controlled environments.

\subsection{Additional Considerations for Testing}

As an additional measure, automated testing is integrated into the development workflow to ensure that unit tests, integration tests, and system tests are executed regularly. This continuous testing approach helps in early detection of issues and maintains the overall quality of the system throughout the development lifecycle.

Due to the nature of mobile and third-party dependencies, certain aspects such as real-time sensor accuracy and external service reliability can only be partially assessed during testing. Where feasible, these aspects are validated through controlled experiments and user feedback during the beta testing phase. 

Developers should include both alpha and beta testing stages. In the alpha stage, the development team and a small group of stakeholders test the application internally to detect and resolve major issues. After that, a beta stage takes place, allowing a broader group of users to try the application in real-world conditions. All testing activities should follow industry best practices and be fully documented to track issues and resolutions with the use of a task management software. 